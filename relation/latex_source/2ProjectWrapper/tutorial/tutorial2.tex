\subsection{Tutorial}
Il tutorial di seguito è possibile anche trovarlo al link: \href{https://github.com/Wabri/ATTSW_Exam/blob/master/gradle.example/second/}{\textsc{github.com/Wabri/ATTSW\_Exam/blob/master/gradle.example/second/}}.
\begin{enumerate}
    \item Creare una cartella gradle.example/second
    \item Eseguire la build:
\begin{verbatim}
    $ gradle init --type java-application
\end{verbatim}
    \item Usare il wrapper per controllare la versione attualmente in uso dal progetto:
\begin{verbatim}
    ./gradlew --version\end{verbatim}
    \item Cambiare la versione del wrapper alla 2.0:
\begin{verbatim}
    $ ./gradlew wrapper --gradle-version 2.0\end{verbatim}
    \item Controllare se la versione del wrapper è stata cambiata:
\begin{verbatim}
    $ ./gradlew --version\end{verbatim}
    \item Fare l'upgrade alla 3.0 del wrapper usando le properties, modificando il campo \texttt{distribuitionUrl}:
\begin{verbatim}
distributionUrl=https\://services.gradle.org/distributions/gradle-3.0-bin.zip \end{verbatim}
    \item Controllare se la versione del wrapper è stata cambiata
    \item Modificare il build.gradle per impostare la versione del wrapper alla 4.6:
\begin{lstlisting}[frame=single]
task wrapper(type: Wrapper) {
    gradleVersion = '4.6'
}
\end{lstlisting}
    \item La versione non sarà modificata fintanto che non sarà eseguito il task \texttt{wrapper}, eseguire quindi la build:
\begin{verbatim}
    $ ./gradlew wrapper\end{verbatim}
    \item Controllare se la versione del wrapper è stata cambiata
    \item Impostare All come tipo di distribuzione da usare per il wrapper, aggiungere quindi il campo \texttt{distributionType}:
\begin{lstlisting}[frame=single]
task  wrapper(type: Wrapper) {
    gradleVersion ='4.6'
    distributionType = Wrapper.DistributionType.ALL
}
\end{lstlisting}
    \item Eseguire la build del task \texttt{wrapper} per aggiornare la distribuzione usata dal wrapper
    \item Visualizzare lo stato attuale dei daemon attualmente in esecuzione:
\begin{verbatim}
    $ ./gradlew --status\end{verbatim}
    \item Eseguire il task \texttt{test} usando il daemon:
\begin{verbatim}
    $ ./gradlew --daemon test\end{verbatim}
    (Notare che se viene rieseguita questa build il tempo di esecuzione risulta essere nullo)
    \item Eseguire lo stesso task precedente senza usare il daemon:
\begin{verbatim}
    $ ./gradlew --no-daemon test\end{verbatim}
    (Notare che non usando il daemon il tempo di esecuzione non sarà nullo)
    \item Stoppare il daemon attivo attualmente usato:
\begin{verbatim}
    $ ./gradlew --stop <PID>\end{verbatim}
\end{enumerate}