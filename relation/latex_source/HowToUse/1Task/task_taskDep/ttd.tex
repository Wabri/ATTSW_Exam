\subsection{Configurazione del build.gradle}
\label{buildGradle}
Come in Maven ci sono i goals, in Gradle ci sono i tasks ognuno dei quali ha il suo scopo definito nella sua implementazione. L'implementazione dei tasks viene fatta in un file di configurazione solitamente nominato build.gradle, che non è altro che uno script in linguaggio Groovy. Creaiamo quindi una cartella in cui inserire la nostra configurazione di gradle e creiamo il file build.gradle in cui andremo a inserire:

\begin{verbatim}
    description = 'Example of Task'

    task dependenceZero {
        description = 'Build Dependence Zero'
        doFirst {
            println 'First Zero'
        }
        doLast {
            println 'Last Zero'
        }
    }

    task dependenceOne(dependsOn: [dependenceZero]) {
        description = 'Build Dependence One'
        doFirst {
            println 'First One'
        }
        doLast {
            println 'Last One'
        }
    }

    task dependenceTwo {
        description = 'Build Dependence Two'
        doFirst {
            println 'First Two'
        }
        doLast {
            println 'Last Two'
        }
    }

    task mainTask(dependsOn: [dependenceOne, dependenceTwo]) {
        description = 'Build Main Task'
        doFirst {
            println 'First MainTask'
        }
        doLast {
            println 'Last MainTask'
        }
    }
\end{verbatim}

In questa build abbiamo definito 4 task: dependenceZero, dependeceOne, dependenceTwo e mainTask. Nella definizione del task può essere usata la parola \textsc{dependsOn} per indicare che il task definito dipende da uno o più task. Nel caso di dependenceOne abbiamo una sola dipendenza che è dependenceZero, nel caso invece di taskMain si hanno 2 dipendenze che sono dependenceOne e dependenceTwo. Possiamo notare che si è data una descrizione sia dei tasks che della build, questo non serve nella pratica ma è buona norma dare sempre una spiegazione sia della build che dei nuovi task che si creano. All'interno dei tasks si nota che ci sono definite delle azioni: doFirst e doLast, quando sarà eseguita la build di un task verrà eseguita prima doFirst e infine doLast. Con la configurazione precedente abbiamo creato un albero delle dipendenze di questo tipo:

\begin{figure}[H]
\includegraphics[scale=0.40]{HowToUse/1Task/task_taskDep/graphDep.png}
\end{figure}

Le builds di gradle vengono eseguite usando il comando da terminale \texttt{\$ gradle taskName}, per l'esempio è possibile quindi eseguire le builds:
\begin{itemize}
    \item \begin{verbatim} $ gradle dependenceZero \end{verbatim}
    \item \begin{verbatim} $ gradle dependenceOne \end{verbatim}
    \item \begin{verbatim} $ gradle dependenceTwo \end{verbatim}
    \item \begin{verbatim} $ gradle mainTask \end{verbatim}
\end{itemize}
Ma è anche possibile eseguire più task contemporaneamente, per esempio:
\begin{itemize}
    \item \begin{verbatim} $ gradle dependenceZero mainTask\end{verbatim}
    \item \begin{verbatim} $ gradle dependenceOne dependenceTwo \end{verbatim}
    \item \begin{verbatim} $ gradle dependenceOne dependenceTwo mainTask\end{verbatim}
\end{itemize}
Considerando che il mainTask è dipendente da dependenceOne e dependenceTwo, l'ultimo esempio non aggiunge niente di più alla build dato che verrebbero comunque eseguiti i 2 tasks. Se eseguiamo infatti \begin{verbatim}$ gradle mainTask \end{verbatim} e poi \begin{verbatim}$ gradle dependenceOne dependenceTwo mainTask\end{verbatim} otterremo il solito output, che è il seguende:
\label{outMainTask}
\begin{verbatim}
> Task :dependenceZero 
First Zero
Last Zero

> Task :dependenceOne 
First One
Last One

> Task :dependenceTwo 
First Two
Last Two

> Task :mainTask 
First MainTask
Last MainTask


BUILD SUCCESSFUL in 0s
4 actionable tasks: 4 executed\end{verbatim} 
