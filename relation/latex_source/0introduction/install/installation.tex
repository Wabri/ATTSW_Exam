(La relazione è stata scritta per SO Linux, ma i concetti generali valgono per tutti)
\subsection{Installazione}
L'istallazione di Gradle può essere fatta in più modi: tramite installazione manuale o utilizzando un package manager (tutte le informazioni possono essere trovate in \textbf{\href{https://gradle.org/install/}{questo link}}). Personalmente consiglio l'utilizzo del software development kit manager \textbf{\href{http://sdkman.io/}{SDKMAN!}} che non solo permette l'installazione molto facilitata di Gradle, ma anche della JVM e di tanti altri tools. L'installazione si basa su 2 semplici comandi:
\begin{verbatim}
  $ curl -s "https://get.sdkman.io" | bash
  
  $ source "$HOME/.sdkman/bin/sdkman-init.sh" \end{verbatim}
A questo punto se tutto è andato a buon fine SDKMAN! è stato installato correttamente, è possibile verificarlo digitando il comando su terminale:
\begin{verbatim}
  $ sdk version \end{verbatim}
l'output risultante dovrebbe essere qualcosa del tipo:
\begin{verbatim}
  SDKMAN 5.5.15+284 \end{verbatim}
Ora è possibile procedere con l'installazione di Gradle. Prima di tutto visualizziamo la lista delle versioni di Gradle:
\begin{verbatim}
  $ sdk list gradle \end{verbatim}
L'output corrispondente sarà:
\begin{verbatim}
================================================================================
Available Gradle Versions
================================================================================
     4.6-rc-2             4.3.1                3.5                  2.2.1          
     4.6-rc-1             4.3-rc-4             3.4.1                2.2            
     4.6                  4.3-rc-3             3.4                  2.14.1         
     4.5.1                4.3-rc-2             3.3                  2.14           
     4.5-rc-2             4.3-rc-1             3.2.1                2.13           
     4.5-rc-1             4.3                  3.2                  2.12           
     4.5                  4.2.1                3.1                  2.11           
     4.4.1                4.2-rc-2             3.0                  2.10           
     4.4-rc-6             4.2-rc-1             2.9                  2.1            
     4.4-rc-5             4.2                  2.8                  2.0            
     4.4-rc-4             4.1                  2.7                  1.9            
     4.4-rc-3             4.0.2                2.6                  1.8            
     4.4-rc-2             4.0.1                2.5                  1.7            
     4.4-rc-1             4.0                  2.4                  1.6            
     4.4                  3.5.1                2.3                  1.5            

================================================================================
+ - local version
* - installed
> - currently in use
================================================================================ \end{verbatim}
La versione che vogliamo installare è quella più recente che in questo caso è la 4.6, possiamo quindi eseguire il comando:
\begin{verbatim}
  $ sdk install gradle 4.6 \end{verbatim}
appena il download e l'installazione sarà finita possiamo verificare il completamento tramite:
\begin{verbatim}
  $ gradle -v \end{verbatim}
che non solo stamperà su terminale la versione di Gradle, ma anche:
\begin{itemize}
  \item \href{http://www.groovy-lang.org/}{Groovy} (linguaggio di programmazione usato per scrivere i file di configurazione)
  \item Ant (software usato per le build delle Java applications)
  \item Java Virtual Machine
  \item sistema operativo in uso
\end{itemize}
se l'output ha queste informazioni allora Gradle è stato completamente installato. SDKMAN! si preoccupa anche di creare la variabile \$GRADLE\_HOME che è possibile visualizzare con il comando 
\begin{verbatim} 
    $ echo $GRADLE_HOME \end{verbatim} 
Se ci sono errori di tipo Java, i problemi possono essere:
\begin{itemize}
  \item Gradle non riesce a trovare la jdk, problema risolvibile installando java con sdkman con il comando 
  \begin{verbatim}
    $ sdk install java <versione>  \end{verbatim}
  \item Java è aggiornato alla versione 9 o superiori (infatti attualmente Gradle non è aggiornato per versioni superiori alla 8), basterà fare un downgrade ad una versione precedente (possibile farlo anche tramite SDKMAN!).
\end{itemize}
In entrambi i casi sarà necessario anche comunicare al sistema la versione da usare: 
\begin{verbatim}  
    $ sdk dafault java <versione_installata> \end{verbatim} 
per essere sicuri che è stata installata la giusta versione di java possiamo controllare gli outputs dei seguenti comandi:
\begin{itemize}
  \item \begin{verbatim} $ echo $JAVA_HOME \end{verbatim}
  \item \begin{verbatim} $ java -version \end{verbatim}
\end{itemize}
il primo comando dovrà restituire in output il giusto percorso della JVM installata, il secondo serve a controllare la versione java attualmente in uso.

