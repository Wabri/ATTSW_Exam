\documentclass{article}

\usepackage[italian]{babel}
\usepackage[utf8]{inputenc}
\usepackage[export]{adjustbox}
\usepackage{wrapfig}
\usepackage{graphicx}
\usepackage{float}
\usepackage{hyperref}
\usepackage{natbib}
\usepackage[margin=2.5cm]{geometry}
\usepackage{minted}
\usepackage{graphicx}
\usepackage{subcaption}
\usepackage{listings}
\usepackage{color}

\definecolor{dkgreen}{rgb}{0,0.6,0}
\definecolor{gray}{rgb}{0.5,0.5,0.5}
\definecolor{mauve}{rgb}{0.58,0,0.82}
\lstset{frame=tb,
  language=Java,
  aboveskip=3mm,
  belowskip=3mm,
  numbers=left,
  columns=flexible,
  basicstyle={\small\ttfamily},
  numberstyle=\tiny\color{gray},
  keywordstyle=\color{blue},
  commentstyle=\color{dkgreen},
  stringstyle=\color{mauve},
  breaklines=true,
  breakatwhitespace=false,
  tabsize=3
}

\title{\textbf{Advanced Techniques and Tools for Software Development: Gradle Build Tool}}
\author{Gabriele Puliti - \texttt{5300140} - \href{mailto:gabriele.puliti@stud.unifi.it}{\textit{gabriele.puliti@stud.unifi.it}}}
\date{Aprile 2018}

\begin{document}

\pagenumbering{Roman}
\maketitle

\newpage
\tableofcontents
\newpage

\begin{flushleft}

\section{Premessa}
Questa relazione è stata scritta per SO Linux e sviluppo software Java usando IDE open source Eclipse. Tutti i concetti con le dovute precisazioni possono essere considerati anche per altri sistemi operativi, altri linguaggi di programmazione e altri IDE. I dati e le informazioni sono state prese dai manuali di Gradle e nei forum relativi:
\begin{itemize}
    \item User manual formato pdf: \textbf{\href{https://docs.gradle.org/current/userguide/userguide.pdf}{docs.gradle.org/current/userguide/userguide.pdf}}
    \item User manual online: \textbf{\href{https://docs.gradle.org/current/userguide/userguide.html}{docs.gradle.org/current/userguide/userguide.htm}}
    \item Forum: \textbf{\href{https://discuss.gradle.org/}{discuss.gradle.org}}
\end{itemize}

\section{Introduzione: Gradle} % https://github.com/gradle/gradle 
Gradle è un progetto open source che fornisce un tool di build automation, che può essere un ottimo sostituto di Maven. Offre un modello in grado di sostenere l'intero ciclo di vita dello sviluppo del software ed è stato progettato per supportare build automation attraverso più linguaggi e piattaforme. Nel nostro caso considereremo questo tool per lo sviluppo di software Java.

\subsection{Differenze tra Gradle e Maven}
Uno dei build tool più usati attualmente è senza dubbio Maven. Ci sono molte differenze tra questi due tools: flessibilità, performance, gestione delle dipendenze e molto altro. Le differenze si possono già notare dal file di configurazione, Gradle infatti ha una convenzione molto più facile e comprensibile rispetto alla tediosa configurazione del così detto pom di Maven. Anche se entrambi usano dei metodi di miglioramento della velocità di esecuzione delle build, Gradle è senza dubbio il tool più veloce. Per essere migliore Grandle usufruisce di:
\begin{itemize}
    \item \textbf{Incrementality:} evitando il lavoro di monitoraggio dei task di I/O eseguendo solo il necessario e quando possibile processare solo i files che sono cambiati;
    \item \textbf{Build Cache:} utilizza un sistema di cache riusando gli outputs di altre build Gradle con gli stessi inputs;
    \item \textbf{Deamon:} sfrutta un long-lived process che mantiene tutte le informazioni in memoria.
\end{itemize}
Queste 3 caratteristiche rendono Gradle molto veloce, ad esempio una build Gradle con Maven verrebbe completata con un tempo 3 volte maggiore. Tutto questo è anche possibile grazie a un sistema di esecuzioni parallele di task e intra-task.
\begin{figure}[H]
\centering
\includegraphics[width=0.7\linewidth]{0introduction/gradle/performance.png}
\end{figure}
Possiamo quindi affermare che Gradle può essere un ottimo sostituto di Maven.

\subsection{Installazione}
L'istallazione di Gradle può essere fatta in più modi: tramite installazione manuale o utilizzando un package manager (tutte le informazioni possono essere trovate in \textbf{\href{https://gradle.org/install/}{questo link}}). Personalmente consiglio l'utilizzo del software development kit manager \textbf{\href{http://sdkman.io/}{SDKMAN!}} che non solo permette l'installazione molto facilitata di Gradle, ma anche della JVM e di tanti altri tools.

\subsubsection{installazione tramite SDKMAN!}
L'installazione si basa su 2 semplici comandi:
\begin{verbatim}
  $ curl -s "https://get.sdkman.io" | bash
  
  $ source "$HOME/.sdkman/bin/sdkman-init.sh" \end{verbatim}
A questo punto se tutto è andato a buon fine SDKMAN! è stato installato correttamente, è possibile verificarlo digitando il comando su terminale:
\begin{verbatim}
  $ sdk version \end{verbatim}
l'output risultante dovrebbe essere qualcosa del tipo:
\begin{verbatim}
  SDKMAN 5.5.15+284 \end{verbatim}
Ora è possibile procedere con l'installazione di Gradle. Prima di tutto visualizziamo la lista delle versioni di Gradle:
\begin{verbatim}
  $ sdk list gradle \end{verbatim}
L'output corrispondente sarà:
\begin{verbatim}
================================================================================
Available Gradle Versions
================================================================================
     4.4.1                4.2-rc-2             3.0                  2.10           
     4.4-rc-6             4.2-rc-1             2.9                  2.1            
     4.4-rc-5             4.2                  2.8                  2.0            
     4.4-rc-4             4.1                  2.7                  1.9            
     4.4-rc-3             4.0.2                2.6                  1.8            
     4.4-rc-2             4.0.1                2.5                  1.7            
     4.4-rc-1             4.0                  2.4                  1.6            
     4.4                  3.5.1                2.3                  1.5            
     4.3.1                3.5                  2.2.1                1.4            
     4.3-rc-4             3.4.1                2.2                  1.3            
     4.3-rc-3             3.4                  2.14.1               1.2            
     4.3-rc-2             3.3                  2.14                 1.12           
     4.3-rc-1             3.2.1                2.13                 1.11           
     4.3                  3.2                  2.12                 1.10           
     4.2.1                3.1                  2.11                 1.1            
================================================================================
+ - local version
* - installed
> - currently in use
================================================================================ \end{verbatim}
La versione che vogliamo installare è quella più recente che in questo caso è la 4.4.1, possiamo quindi eseguire il comando:
\begin{verbatim}
  $ sdk install gradle 4.4.1 \end{verbatim}
appena il download e l'installazione sarà finita possiamo verificare il completamento tramite:
\begin{verbatim}
  $ gradle -v \end{verbatim}
che non solo stamperà su terminale la versione di Gradle, ma anche:
\begin{itemize}
  \item \href{http://www.groovy-lang.org/}{Groovy} (linguaggio di programmazione usato per scrivere i file di configurazione)
  \item Ant (software usato per le build delle Java applications)
  \item Java Virtual Machine
  \item sistema operativo in uso
\end{itemize}
se l'output ha queste informazioni allora Gradle è stato completamente installato. SDKMAN! si preoccupa anche di creare la variabile \$GRADLE\_HOME che è possibile visualizzare con il comando 
\begin{verbatim} 
    $ echo $GRADLE_HOME \end{verbatim} 
Se ci sono errori di tipo Java, i problemi possono essere:
\begin{itemize}
  \item Gradle non riesce a trovare la jdk, problema risolvibile installando java con sdkman con il comando 
  \begin{verbatim}
    $ sdk install java <versione>  \end{verbatim}
  \item Java è aggiornato alla versione 9 o superiori (infatti attualmente Gradle 4.4.1 non è aggiornato per versioni superiori alla 8), basterà fare un downgrade ad una versione precedente (possibile farlo anche tramite SDKMAN!).
\end{itemize}
In entrambi i casi sarà necessario anche comunicare al sistema la versione da usare: 
\begin{verbatim}  
    $ sdk dafault java <versione_installata> \end{verbatim} 
per essere sicuri che è stata installata la giusta versione di java possiamo controllare gli outputs dei seguenti comandi:
\begin{itemize}
  \item \begin{verbatim} $ echo $JAVA_HOME \end{verbatim}
  \item \begin{verbatim} $ java -version \end{verbatim}
\end{itemize}
il primo comando dovrà restituire in output il giusto percorso della JVM installata, il secondo serve a controllare la versione java attualmente in uso.


\newpage

\pagenumbering{arabic}
\section{Tasks \& Task Dependencies}

\subsection{Configurazione del build.gradle}
\label{buildGradle}
Come in Maven ci sono i goals, in Gradle ci sono i tasks ognuno dei quali ha il suo scopo definito nella sua implementazione. L'implementazione dei tasks viene fatta in un file di configurazione solitamente nominato build.gradle, che non è altro che uno script in linguaggio Groovy. Creaiamo quindi una cartella in cui inserire la nostra configurazione di gradle e creiamo il file build.gradle in cui andremo a inserire:

\begin{verbatim}
    description = 'Example of Task'

    task dependenceZero {
        description = 'Build Dependence Zero'
        doFirst {
            println 'First Zero'
        }
        doLast {
            println 'Last Zero'
        }
    }

    task dependenceOne(dependsOn: [dependenceZero]) {
        description = 'Build Dependence One'
        doFirst {
            println 'First One'
        }
        doLast {
            println 'Last One'
        }
    }

    task dependenceTwo {
        description = 'Build Dependence Two'
        doFirst {
            println 'First Two'
        }
        doLast {
            println 'Last Two'
        }
    }

    task mainTask(dependsOn: [dependenceOne, dependenceTwo]) {
        description = 'Build Main Task'
        doFirst {
            println 'First MainTask'
        }
        doLast {
            println 'Last MainTask'
        }
    }
\end{verbatim}

In questa build abbiamo definito 4 task: dependenceZero, dependeceOne, dependenceTwo e mainTask. Nella definizione del task può essere usata la parola \textsc{dependsOn} per indicare che il task definito dipende da uno o più task. Nel caso di dependenceOne abbiamo una sola dipendenza che è dependenceZero, nel caso invece di taskMain si hanno 2 dipendenze che sono dependenceOne e dependenceTwo. Possiamo notare che si è data una descrizione sia dei tasks che della build, questo non serve nella pratica ma è buona norma dare sempre una spiegazione sia della build che dei nuovi task che si creano. All'interno dei tasks si nota che ci sono definite delle azioni: doFirst e doLast, quando sarà eseguita la build di un task verrà eseguita prima doFirst e infine doLast. Con la configurazione precedente abbiamo creato un albero delle dipendenze di questo tipo:

\begin{figure}[H]
\includegraphics[scale=0.40]{HowToUse/1Task/task_taskDep/graphDep.png}
\end{figure}

Le builds di gradle vengono eseguite usando il comando da terminale \texttt{\$ gradle taskName}, per l'esempio è possibile quindi eseguire le builds:
\begin{itemize}
    \item \begin{verbatim} $ gradle dependenceZero \end{verbatim}
    \item \begin{verbatim} $ gradle dependenceOne \end{verbatim}
    \item \begin{verbatim} $ gradle dependenceTwo \end{verbatim}
    \item \begin{verbatim} $ gradle mainTask \end{verbatim}
\end{itemize}
Ma è anche possibile eseguire più task contemporaneamente, per esempio:
\begin{itemize}
    \item \begin{verbatim} $ gradle dependenceZero mainTask\end{verbatim}
    \item \begin{verbatim} $ gradle dependenceOne dependenceTwo \end{verbatim}
    \item \begin{verbatim} $ gradle dependenceOne dependenceTwo mainTask\end{verbatim}
\end{itemize}
Considerando che il mainTask è dipendente da dependenceOne e dependenceTwo, l'ultimo esempio non aggiunge niente di più alla build dato che verrebbero comunque eseguiti i 2 tasks. Se eseguiamo infatti \begin{verbatim}$ gradle mainTask \end{verbatim} e poi \begin{verbatim}$ gradle dependenceOne dependenceTwo mainTask\end{verbatim} otterremo il solito output, che è il seguende:
\label{outMainTask}
\begin{verbatim}
> Task :dependenceZero 
First Zero
Last Zero

> Task :dependenceOne 
First One
Last One

> Task :dependenceTwo 
First Two
Last Two

> Task :mainTask 
First MainTask
Last MainTask


BUILD SUCCESSFUL in 0s
4 actionable tasks: 4 executed\end{verbatim} 


\subsection{Approfondimenti}
Andiamo ad approfondire le azioni che è possibile fare tramite il terminale.
\subsubsection{Abbreviazione dei nomi}
È possibile abbreviare il nome del task da eseguire stando però attenti ad identificarlo unicamente, per esempio se volessi eseguire il task \textbf{dependenceTwo} potrei farlo semplicemente con il comando:
\begin{verbatim}
    $ gradle depTw \end{verbatim}
considerando i task creati precedentemente notiamo che il task è univocamente identificato.

\subsubsection{Escludere i task}
È possibile escludere un task di una build, aggiungendo come argomento il task da escludere preceduto da -x:
\begin{verbatim}
    $ gradle <task_da_eseguire> -x <task_da_escludere> \end{verbatim}
questo viene usato al fine di eliminare un task inutile per lo scopo della build che abbiamo intenzione di eseguire. Riprendendo l'output di \begin{verbatim}   $ gradle mainTask \end{verbatim} notiamo che vengono eseguiti tutti i tasks definiti nella build.gradle (a pagina \pageref{outMainTask}), se volessimo escludere dependenceOne dalla build allora dovremo eseguire:
\begin{verbatim}
    $ gradle mainTask -x dependenceOne \end{verbatim}
Otteniamo in questo modo in output:
\begin{verbatim}
> Task :dependenceTwo 
First Two
Last Two

> Task :mainTask 
First MainTask
Last MainTask


BUILD SUCCESSFUL in 0s
2 actionable tasks: 2 executed
\end{verbatim}
Possiamo notare che non verrà eseguito nemmeno il task dependenceZero perchè è una dipendenza del task dependenceOne.

\subsubsection{Selezionare la build da eseguire}
Consideriamo che esista in una subdirectory chiamata subdir una build chiamata subbild.gradle, partendo dalla directory source è possibile eseguire questa build eseguendo il comando:
\begin{verbatim}
    $ gradle -b subdir/subbuild.gradle <task_da_eseguire> \end{verbatim}
Questa particolare funzione serve soprattutto ai progetti multi-builds, in cui è necessario avere a disposizione più di una build di riferimento.

\subsubsection{Forzare l'esecuzione di un task} 
A causa della Gradle cache è possibile che un task o più di uno non vengano eseguiti perchè marcati come UP-TO-DATE (anche se dalla versione Gradle 4.0 non viene più mostrato in output), in questo caso è possibile forzarne l'esecuzione con:
\begin{verbatim}
    $ gradle --rerun-tasks <tasks_da_eseguire> \end{verbatim}
    
\subsubsection{Continuare la build quando si verifica un errore}
Se durante una build un task fallisce, Gradle di default interromperà l'esecuzione e farà fallire anche la build. Questo permette alla build di completare velocemente, ma il fallimento anticipato della build potrebbe nascondere altri problemi che possono presentarsi in altri tasks. A volte è quindi necessario imporre ad una build di gradle di continuare nonostante il fallimento di uno o più tasks, questo è possibile usando l'opzione \texttt{--continue}:
\begin{verbatim}
    $ gradle <tasks_da_eseguire> --continue\end{verbatim}
In questo modo verranno eseguiti tutti i tasks e solo al completamento della build saranno resi noti gli errori.

\subsubsection{Ottenere informazioni generali} 
Per visualizzare una lista dei principali tasks eseguibili è possibile eseguire il task \begin{verbatim}    $ gradle tasks\end{verbatim} l'output di questa build sarà:
\begin{verbatim}
> Task :tasks 

------------------------------------------------------------
All tasks runnable from root project - Example of Task
------------------------------------------------------------

Build Setup tasks
-----------------
init - Initializes a new Gradle build.
wrapper - Generates Gradle wrapper files.

Help tasks
----------
buildEnvironment - Displays all buildscript dependencies declared in root project 'src'.
components - Displays the components produced by root project 'src'. [incubating]
dependencies - Displays all dependencies declared in root project 'src'.
dependencyInsight - Displays the insight into a specific dependency in root project 'src'.
dependentComponents - Displays the dependent components of components in root project 'src'. 
[incubating]
help - Displays a help message.
model - Displays the configuration model of root project 'src'. [incubating]
projects - Displays the sub-projects of root project 'src'.
properties - Displays the properties of root project 'src'.
tasks - Displays the tasks runnable from root project 'src'.

To see all tasks and more detail, run gradle tasks --all

To see more detail about a task, run gradle help --task <task>


BUILD SUCCESSFUL in 0s
1 actionable task: 1 executed
\end{verbatim}

come dice l'output, per visualizzare la lista di tutti i tasks eseguibili nel nostro project è necessario eseguire la build del task \begin{verbatim}$ gradle tasks --all \end{verbatim} noteremo che in questo caso verranno visualizzati anche i tasks che abbiamo precedentemente creato (dependenceZero, dependenceOne, dependenceTwo, mainTask con le relative descrizioni):
\begin{verbatim}
> Task :tasks 

------------------------------------------------------------
All tasks runnable from root project - Example of Task
------------------------------------------------------------

Build Setup tasks
-----------------
init - Initializes a new Gradle build.
wrapper - Generates Gradle wrapper files.

Help tasks
----------
buildEnvironment - Displays all buildscript dependencies declared in root project 'src'.
components - Displays the components produced by root project 'src'. [incubating]
dependencies - Displays all dependencies declared in root project 'src'.
dependencyInsight - Displays the insight into a specific dependency in root project 'src'.
dependentComponents - Displays the dependent components of components in root project 'src'. 
[incubating]
help - Displays a help message.
model - Displays the configuration model of root project 'src'. [incubating]
projects - Displays the sub-projects of root project 'src'.
properties - Displays the properties of root project 'src'.
tasks - Displays the tasks runnable from root project 'src'.

Other tasks
-----------
dependenceOne - Build Dependence One
dependenceTwo - Build Dependence Two
dependenceZero - Build Dependence Zero
mainTask - Build Main Task


BUILD SUCCESSFUL in 0s
1 actionable task: 1 executed
\end{verbatim}
Se invece vogliamo informazioni più specifiche riguardo un singolo task la build da fare è 
\begin{verbatim}
    $ gradle help --task <nome_del_task>\end{verbatim}
per esempio eseguiamo:
\begin{verbatim}
    $ gradle help --task mainTask\end{verbatim}
otterremo una descrizione specifica del task mainTask:
\begin{verbatim}
> Task :help 
Detailed task information for mainTask

Path
     :mainTask

Type
     Task (org.gradle.api.Task)

Description
     Build Main Task

Group
     -\end{verbatim}

\subsubsection{Build scan}
Una funzione molto interessante di Gradle è la possibilità di poter pubblicare la propria build, questo permette di avere un report completo e condivisibile. Per utilizzare questa funzionalità è necessario aggiungere alla build di un task l'opzione \texttt{--scan}:
\begin{verbatim}    $ gradle <task_da_eseguire> --scan \end{verbatim}
Al completamento della build del task verrà richiesto di accettare i termini di uso di questo servizio. Una volta accettati verrà fornito un link alla build pubblicata in cui sarà richiesta una mail di riferimento per confermare la pubblicazione della build. Prendendo come esempio eseguiamo il comando:
\begin{verbatim}
    $ gradle mainTask --scan\end{verbatim}
l'output risultante sarà:
\begin{verbatim}
> Task :dependenceZero 
First Zero
Last Zero

> Task :dependenceOne 
First One
Last One

> Task :dependenceTwo 
First Two
Last Two

> Task :mainTask 
First MainTask
Last MainTask


BUILD SUCCESSFUL in 1s
4 actionable tasks: 4 executed

Publishing a build scan to scans.gradle.com requires accepting the Gradle Terms of Service 
defined at https://gradle.com/terms-of-service. Do you accept these terms? [yes, no] 
yes
Gradle Terms of Service accepted.

Publishing build scan...
https://scans.gradle.com/s/qcc4vkuegibig
\end{verbatim}
cliccando sul sito e seguendo le indicazioni, il risultato finale sarà:
\begin{figure}[H]
\includegraphics[scale=0.25]{1Task/insights/gradleScan.png}
\end{figure}

\subsection{Tutorial}
Il tutorial di seguito è possibile anche trovarlo al link: \href{https://github.com/Wabri/ATTSW_Exam/blob/master/gradle.example/first/}{\textsc{github.com/Wabri/ATTSW\_Exam/blob/master/gradle.example/first/}}.
\begin{enumerate}
    \item Creare una cartella gradle.example/first
    \item All'interno della nuova cartella creare il file build.gradle contenente:
\begin{lstlisting}[frame=single]
description = 'Example of Task'

task dependenceZero {
	description = 'Build Dependence Zero'
	doFirst {
		println 'First Zero'
	}
	doLast {
		println 'Last Zero'
	}
}

task dependenceOne(dependsOn: [dependenceZero]) {
	description = 'Build Dependence One'
	doFirst {
		println 'First One'
	}
	doLast {
		println 'Last One'
	}
}

task dependenceTwo {
	description = 'Build Dependence Two'
	doFirst {
		println 'First Two'
	}
	doLast {
		println 'Last Two'
	}
}

task mainTask(dependsOn: [dependenceOne, dependenceTwo]) {
	description = 'Build Main Task'
	doFirst {
		println 'First MainTask'
	}
	doLast {
		println 'Last MainTask'
	}
}
\end{lstlisting}
    \item Eseguire la build:
\begin{verbatim}
    $ gradle mainTask
\end{verbatim}
    \item  Eseguire la build multi-tasks:
\begin{verbatim}
    $ gradle dependenceZero dependenceTwo
\end{verbatim}
    \item Eseguire la build usando una abbreviazione:
\begin{verbatim}
    $ gradle maTa
\end{verbatim}
    \item Eseguire la build precedente escludendo il task dependenceOne:
\begin{verbatim}
    $ gradle mainTask -x dependenceOne
\end{verbatim}
    \item Creare una build differente in una subdirectory rispetto alla posizione iniziale: 
\begin{lstlisting}[frame=single]
description = 'Sub directory'

task subMainTask {
	description = 'Sub Build Main Task'
	doFirst {
		println 'First MainTask'
	}
	doLast {
		println 'Last MainTask'
	}
}
\end{lstlisting}
    \item Eseguire il task subMainTask della build appena creata partendo dalla directory root:
\begin{verbatim}
    $ gradle -b subdir/build.gradle suMT        
\end{verbatim}
    \item Forzare l'esecuzione di un task marcato come UP-TO-DATE:
\begin{verbatim}
    $ gradle --rerun-tasks maTa
\end{verbatim}
    \item Ottenere la lista dei tasks di default:
\begin{verbatim}
    $ gradle tasks
\end{verbatim}
    \item Ottenere la lista di tutti i tasks:
\begin{verbatim}
    $ gradle tasks --all
\end{verbatim}
    \item Eseguire il comando:
\begin{verbatim}
    $ gradle help --task mainTask
\end{verbatim}
    \item Pubblicare la build del task mainTask:
\begin{verbatim}
    $ gradle mainTask --scan
\end{verbatim}
\end{enumerate}
\newpage
\section{Project, Wrapper \& Deamon}
Potete trovare un tutorial guidato a questo link: \href{https://github.com/Wabri/ATTSW_Exam/tree/master/gradle.example/second}{\textsc{github.com/Wabri/ATTSW\_Exam/tree/master/gradle.example/second}}

\subsection{Creazione di un nuovo progetto Gradle}
Creare un progetto Gradle è molto semplice sfruttando direttamente il task di default \textsc{init}. Creiamo una cartella in cui eseguiremo da terminale il comando:
\begin{verbatim}
    $ gradle init\end{verbatim}
Notiamo che nella directory sono stati creati 4 file e 1 cartella:
\begin{itemize}
    \item \texttt{build.gradle} e \texttt{settings.gradle}: sono i file di configurazione
    \item \texttt{gradlew}, \texttt{gradlew.bat} e la directory \texttt{gradle}: sono i file corrispondenti al wrapper
\end{itemize}
Abbiamo già trattato il file di configurazione build.gradle (pagina \pageref{buildGradle}). Il file settings.gradle è anch'esso uno script Groovy dove vengono indicati quali progetti parteciperanno alla build. Questo task crea un progetto di default, ma è possibile essere più specifici in quanto il task \textsc{init} con l'opzione \texttt{--type} può assumere come argomento una tipologia di progetto. Assumiamo che il progetto che vogliamo creare sia una applicazione java, il comando da eseguire sarà:
\begin{verbatim}
    $ gradle init --type java-application\end{verbatim}
Rispetto al comando precedente verrà creata una directory \texttt{src} in più che sarà specifica per il linguaggio java:
\begin{itemize}
    \item \texttt{main/java/App.java}: che è un file preimpostato il cui contenutò sarà un semplice main che stamperà il consueto \textsc{Hello World!}
    \item \texttt{test/java/AppTest.java}: in cui viene testato il metodo contenuto nella classe App.java
\end{itemize}
Spieghiamo a questo punto i file del wrapper precedentemente indicati.


\subsection{Wrapper}
Molto spesso prima di poter usufruire di uno strumento di sviluppo è necessaria una installazione. Gradle mette a disposizione uno script che permette di usare tutte le sue funzionalità evitando di installare Gradle su tutte le macchine di sviluppo, questo strumento viene chiamato Gradle Wrapper. Se in un progetto è stato settato il Wrapper è possibile eseguire le builds sostituendo il comando \texttt{gradle} con il comando \texttt{./gradlew} (se si lavora con sistema operativo windows il comando è \texttt{./gradlew.bat}). Se più persone lavorano a un progetto può capitare che ci siano differenze tra le versioni di uno strumento, nel caso del wrapper non è possibile sbagliare perchè la sua versione è insita durante la sua creazione o durante il suo upgrade (o downgrade). Quindi è sempre consigliato l'uso del wrapper e lasciare tutte le sue informazioni anche nella repository del VCS usato. Per creare il wrapper in un progetto è necessario eseguire il comando:
\begin{verbatim}
    $ gradle wrapper\end{verbatim}
Il comando creerà 4 files:
\begin{itemize}
    \item \textbf{gradlew}: script shell per eseguire il wrapper in sistemi Unix
    \item \textbf{gradlew.bat}: file batch per eseguire il wrapper in sistemi Windows
    \item \textbf{gradle/wrapper/gradle-wrapper.properties}: file di configurazione delle proprietà del Wrapper
    \item \textbf{gradle/wrapper/gradle-wrapper.jar}: contiene il codice effettivo per eseguire le build Gradle
\end{itemize}
Questi sono i file di cui ha bisogno il wrapper per poter essere usato. Ovviamente Quando il wrapper viene creato la sua versione sarà quella di Gradle attualmente installato sulla macchina, è possibile specificare in vari modi quale versione usare:
\begin{enumerate}
    \item eseguire il solito comando con l'aggiunta dell'argomento \texttt{--gradle-version} con il numero della versione:
\begin{verbatim}
    $ gradle wrapper --gradle-version <numero_versione> \end{verbatim}
oppure se già inserito il wrapper:
\begin{verbatim}
    $ ./gradlew wrapper --gradle-version <numero_versione> \end{verbatim}
per esempio se volessimo passare dalla versione attuale alla versione 2.0 basterà eseguire il comando:
\begin{verbatim}
    $ ./gradlew wrapper --gradle-version 2.0 \end{verbatim}
dopo aver eseguito il download della versione, l'output corrispondente sarà:
\begin{verbatim}
------------------------------------------------------------
Gradle 2.0
------------------------------------------------------------
Build time:   2014-07-01 07:45:34 UTC
Build number: none
Revision:     b6ead6fa452dfdadec484059191eb641d817226c
Groovy:       2.3.3
Ant:          Apache Ant(TM) version 1.9.3 compiled on December 23 2013
JVM:          1.8.0_161 (Oracle Corporation 25.161-b12)
OS:           Linux 4.13.0-37-generic amd64
\end{verbatim}
(il task wrapper non esisteva fino alla versione 3.0, eseguire quindi questo task con versioni precedenti risulterebbe in un fallimento della build).
    \item modificare direttamente il file gradle-wrapper.properties in cui ci sarà:
\begin{verbatim}
    distributionUrl=https\://services.gradle.org/distributions/gradle-4.4.1-bin.zip \end{verbatim}
    che è il tipo di distribuzione usata attualmente dal wrapper. Per passare alla versione 2.0 possiamo modificare questa riga con:
\begin{verbatim}
    distributionUrl=https\://services.gradle.org/distributions/gradle-2.0-bin.zip \end{verbatim}
    eseguendo poi un qualsiasi comando la versione sarà aggiornata.
    \item infine è possibile specificarlo direttamente modificando il file build.gradle aggiungendo un task chiamato wrapper che estenderà la classe Wrapper:
\begin{verbatim}
    task wrapper(type: Wrapper) {
        gradleVersion = '2.0'
    } \end{verbatim}
    in questo modo viene effettivamente fatto un override del task wrapper. A questo punto per aggiornare alla versione indicata basterà eseguire il comando:
\begin{verbatim}
    $ ./gradlew wrapper \end{verbatim}
    
\end{enumerate}
In ogni caso possiamo visualizzare la versione usata dal wrapper con il comando:
\begin{verbatim}
    $ ./gradlew --version \end{verbatim}
Il wrapper è altamente configurabile sia come proprietà sia come versionamento. Per esempio riprendendo il punto 3 della lista precedente, se non si vuole specificare tutte le volte il tipo di distribuzione voluta è possibile inserire un altro campo all’interno del task wrapper \texttt{distributionType} a cui assegneremo \texttt{Wrapper.DistributionType.ALL}:
\begin{lstlisting}[frame=single]
task wrapper(type: Wrapper) {
    gradleVersion = '4.6'
    distributionType = Wrapper.DistributionType.ALL
}
\end{lstlisting}
In questo modo verrà scaricata tutta la distribuzione e non solo i file binari.

\subsection{Deamon}
Gradle viene eseguito sulla Java virtual machine (JVM) e usa librerie di supporto che necessitano una inizializzazione, entrambi allungano il processo di esecuzione iniziale della build allungando i tempi di attesa. Questo problema viene risolto usando un Daemon che mantiene le informazioni della build in background velocizzando le esecuzioni successive alla prima, mantenendo le informazioni in memoria pronte all'uso. Una build di gradle è possibile eseguirla con o senza il daemon, indicando nelle proprietà di Gradle quando usarlo e se usarlo. Il daemon permette non solo di evitare l'avviamento della JVM, ma ha anche un sistema di cache in cui sono immagazzinati: struttura del progetto, files, tasks e molto altro. Ovviamente se il progetto viene eseguito in contenitori temporanei, tipo un server di continuous integration (CI), è sconsigliato l'uso del daemon in quanto questi non riutilizzano lo stesso processo ma ne creano uno nuovo, quindi l'uso del deamon non solo è inutile ma ridurrà anche le prestazioni dato che dovrà rieseguire il daemon di gradle e ricreare la cache. Per controllare i processi daemon attivi sulla macchina basterà eseguire il comando:
\begin{verbatim}
    $ gradle --status \end{verbatim}
che restituirà il pid, lo stato e la versione di gradle usata dal deamon. Ad esempio se abbiamo un progetto in cui è usata la versione di gradle 3.0 avremo un risultato di questo tipo:
\begin{verbatim}
    PID STATUS   INFO
  16463 IDLE     3.0 \end{verbatim}
Come già detto è possibile disabilitare il processo deamon, per farlo è necessario modificare il campo \texttt{org.gradle.deamon} con l'assegnazione a false. Le proprietà si trovano seguendo il percorso \texttt{\$HOME/.gradle/gradle.properties}, se il file non esiste basterà crearlo, a questo punto inseriamo in coda al file:
\begin{lstlisting}[frame=single]
    org.gradle.deamon=false
\end{lstlisting}
Ora tutte le volte che eseguiamo una build non verrà riattivato il deamon. Esistono 2 opzioni che permettono di indicare se usare o no il daemon specificatamente su una build:
\begin{itemize}
    \item \texttt{--no-daemon}, che indica di non usare il daemon per questa build
    \item \texttt{--daemon}, che invece indica di usare il daemon per questa build
\end{itemize}
Spesso viene usato questo metodo che risulta essere molto più chiaro soprattutto se la build viene condivisa. Se volessimo stoppare un daemon attivo è possibile farlo con l'opzione \texttt{--stop} seguito dal PID del daemon da stoppare, per esempio se volessimo stoppare un daemon con PID=12812 eseguiremo il comando:
\begin{verbatim}
    $ gradle --stop 12812\end{verbatim}
Il risultato sarà:
\begin{verbatim}
    Stopping Daemon(s)
    1 Daemon stopped\end{verbatim}
Se abbiamo più daemon attivi e volessimo stopparli tutti è possibile eseguire il comando:
\begin{verbatim}
    $ gradle --stop \end{verbatim}
Evitando quindi di eseguire una cascata di comandi tutti uguali con PID diverso. Questo stopperà tutti i daemon attivi aventi la stessa versione di gradle usato per eseguire il comando.

\subsection{Tutorial}
Il tutorial di seguito è possibile anche trovarlo al link: \href{https://github.com/Wabri/ATTSW_Exam/blob/master/gradle.example/second/}{\textsc{github.com/Wabri/ATTSW\_Exam/blob/master/gradle.example/second/}}.
\begin{enumerate}
    \item Creare una cartella gradle.example/second
    \item Eseguire la build:
\begin{verbatim}
    $ gradle init --type java-application
\end{verbatim}
    \item Usare il wrapper per controllare la versione attualmente in uso dal progetto:
\begin{verbatim}
    ./gradlew --version\end{verbatim}
    \item Cambiare la versione del wrapper alla 2.0:
\begin{verbatim}
    $ ./gradlew wrapper --gradle-version 2.0\end{verbatim}
    \item Controllare se la versione del wrapper è stata cambiata:
\begin{verbatim}
    $ ./gradlew --version\end{verbatim}
    \item Fare l'upgrade alla 3.0 del wrapper usando le properties, modificando il campo \texttt{distribuitionUrl}:
\begin{verbatim}
distributionUrl=https\://services.gradle.org/distributions/gradle-3.0-bin.zip \end{verbatim}
    \item Controllare se la versione del wrapper è stata cambiata
    \item Modificare il build.gradle per impostare la versione del wrapper alla 4.6:
\begin{lstlisting}[frame=single]
task wrapper(type: Wrapper) {
    gradleVersion = '4.6'
}
\end{lstlisting}
    \item La versione non sarà modificata fintanto che non sarà eseguito il task \texttt{wrapper}, eseguire quindi la build:
\begin{verbatim}
    $ ./gradlew wrapper\end{verbatim}
    \item Controllare se la versione del wrapper è stata cambiata
    \item Impostare All come tipo di distribuzione da usare per il wrapper, aggiungere quindi il campo \texttt{distributionType}:
\begin{lstlisting}[frame=single]
task  wrapper(type: Wrapper) {
    gradleVersion ='4.6'
    distributionType = Wrapper.DistributionType.ALL
}
\end{lstlisting}
\end{enumerate}
\newpage
\section{Dependency Management}
Una delle parti più importanti di uno strumento di questo tipo è la gestione delle dipendenze che si divide in 2 parti: incoming files e outgoing files. Gradle ha bisogno di conoscere di cosa il nostro progetto ha bisogno per poter essere compilato ed eseguito le così dette dipendenze (dipendencies) che in questo caso sono gli incoming files. Gli outgoing files sono invece tutto ciò che il progetto produce, definite pubblicazioni (pubblications). Le dipendenze vengono specificate in forma di modules, è quindi necessario indicare dove si trovano questi modules in modo tale che Gradle li possa scaricare e impostare per il progetto che stiamo sviluppando. La posizione dove è possibile trovare i modules è definita repository, è necessario quindi dichiarare le repositories usate per le dipendenze volute. Ci sono 2 tipi di repository: 
\begin{enumerate}
    \item esterne, in questo caso la repository si trova in un server online adibito alla raccolta di modules
    \item interne, la repository è una cartella locale al progetto
\end{enumerate}
Per quanto riguarda quelle esterne Gradle si occuperà di scaricarle. E' possibile che alcune dipendenze vengano usate in più progetti, gradle mantiene quindi una cache locale, chiamata dependency cache, in cui salverà i modules già scaricati in modo da evitare di effettuare il download ad ogni build. Possiamo quindi immaginare che il ciclo di risoluzione delle dipendenze esterne sarà questo:
\begin{enumerate}
    \item ricerca delle dipendenze esterne nella cache locale
    \item se non si trovano nella cache locale si controlla se esistono nelle repository specificate
    \item se sono state trovate, vengono scaricate e inserite nella cache locale
\end{enumerate}
Nell'immagine è possibile vedere il percorso specifico che effettua la dependency resolution di Gradle:
\begin{figure}[H]
\centering
\includegraphics[width=0.4\linewidth]{HowToUse/3DependencyManagement/depMan.png}
\end{figure}
Analizziamo ora il caso di un progetto java.

\subsection{Dichiarazione delle dipendenze}
Prima di tutto è necessario indicare in che linguaggio il nostro progetto viene rilasciato (consideriamo d'ora in poi solo il caso di Java), per farlo aggiungiamo in testa al file build.gradle:
\begin{verbatim}
apply plugin: 'java' \end{verbatim}
A questo punto per poter usufruire di una dipendenza è necessario specificare da dove Gradle deve andare a prenderla, dobbiamo quindi indicare il repository remoto di riferimento. Se per esempio vogliamo che il nostro repository di riferimento sia Maven allora dobbiamo aggiungere al build.gradle:
\begin{verbatim}
repositories {
    mavenCentral()
} \end{verbatim}
In questo modo tutte le dipendenze che andremo a indicare successivamente saranno riferimenti alle pubblicazioni su Maven Central. La dichiarazione delle dipendenze deve essere inserita nel tag \texttt{dependencies} nel build.gradle file. Per esempio vogliamo avere junit 4.12 come dipendenza al nostro progetto Gradle allora dobbiamo aggiungere:
\begin{verbatim}
dependencies {
    testCompile group: 'junit', name: 'junit', version: '4.12' 
} \end{verbatim}
Osserviamo che nella dichiarazione ci sono 4 diversi indicatori:
\begin{itemize}
    \item \texttt{testCompile} indica lo scopo della dipendenza, in questo caso sarà importata durante la compilazione dei test;
    \item \texttt{group, name, version} corrispondono rispettivamente al groupId (nome del team o della società che ha sviluppato il modulo), artifactId (nome effettivo del modulo) e al version (versione del modulo) definiti su Maven.
\end{itemize}
Esiste un modo molto più diretto per indicare una dipendenza:
\begin{verbatim}
dependencies {
    testCompile 'junit:junit:4.12'
} \end{verbatim}
Ha lo stesso significato precedente ma ha una forma più compatta, forma che adotta anche la documentazione Maven.
\begin{figure}[H]
\centering
\includegraphics[width=0.4\linewidth]{HowToUse/3DependencyManagement/javaDep/gradleInMavenRepo.png}
\end{figure} 
Possiamo notare ora la differenza sostanziale della configurazione delle dipendenze tra il pom.xml di Maven e la build.gradle di Gradle. A questo punto per scaricare le dipendenze si deve eseguire il comando \texttt{dependencies} il cui output restituirà una lista di tutti i task con le relative dipendenze associate.
\newpage
\section{Integrazione con altri Tools}

\subsection{Continuous Integration with Travis \& Github}
Travis CI (continuous integration) è una piattaforma che mette a disposizione gratuitamente un tool di build automation per progetti open source. Per poter integrare questo tool è necessario avere a disposizione una repository GitHub (\textbf{\href{https://github.com}{github.com}}) dove deve essere ospitato il progetto che stiamo sviluppando. Per maggiori informazioni è possibile leggere la documentazione relativa direttamente nel sito: \textbf{\href{https://travis-ci.org}{travis-ci.org}}.

\subsection{Docker}
Docker è una piattaforma open source per lo sviluppo, il rilascio e l'esecuzione di applicazioni. Questo tool permette di impacchettare ed eseguire applicazioni in un ambiente isolato e sicuro, chiamato container, consentendo di eseguire più contenitori contemporaneamente su una stessa macchina. I container sono molto leggeri dato che non necessitano di un caricamento eccessivo di risorse dato che vengono eseguiti direttamente nel kernel della macchina che li ospita. Docker mette a disposizione tutti gli strumenti e la piattaforma per gestire:
\begin{itemize}
    \item lo sviluppo dell'applicazione e dei suoi compontenti di supporto
    \item la distribuzione e il test dell'applicazione
    \item quando l'applicazione è completata permette di distribuirla tramite contenitore o servizio (che sia locale o in cloud)
\end{itemize}

\subsection{Sonarqube}
SonarQube è uno strumento per analizzare e tenere traccia della qualità del codice di un progetto. Copre principalmente 7 rami che definiscono la qualità del codice:
\begin{itemize}
    \item Architettura e design
    \item Duplicazioni
    \item Unit Test
    \item Complessità
    \item Potenziali bugs
    \item Coding rules
    \item Commenti
\end{itemize}
Analizzando questi punti è in grado di fornire il debito tecnico del codice, cioè il tempo materiale per risolvere tutti i problemi del codice. Per maggiori informazioni è possibile visitare il sito: \textbf{\href{https://www.sonarqube.org/}{www.sonarqube.org}}.

\subsection{Jacoco}
JaCoCo (diminutivo di Java Code Coverage Tools) è un insieme di strumenti open source che mette a disposizione un metodo per misurare il code coverage di un progetto java e rilasciare un report completo. Il report fornisce informazioni di code coverage rispetto a: linee di codice, istruzioni e branch. Per maggiori informazioni: \textbf{\href{http://www.jacoco.org/jacoco/}{www.jacoco.org/jacoco}} oppure \textbf{\href{https://github.com/jacoco/jacoco}{github.com/jacoco/jacoco}}.

\subsection{Tutorial}
Il tutorial di seguito è possibile anche trovarlo al link: \href{https://github.com/Wabri/ATTSW_integrationWithOtherTools/tree/master}{\texttt{github.com/Wabri/ATTSW\_integrationWithOtherTools}}.
\begin{enumerate}
    \item Creare una repository su github e clonarla localmente:
    \begin{figure}[H]
    \centering
    \includegraphics[width=1\linewidth]{4IntegrationWithOtherTool/tutorial/githubRepo.png}
    \end{figure}
    \item Nella repository locale creare una cartella chiamata tutorial:
    \begin{figure}[H]
    \centering
    \includegraphics[width=0.7\linewidth]{4IntegrationWithOtherTool/tutorial/localGithubRepo.png}
    \end{figure}
    \item Spostarsi nella cartella appena creata e eseguire la build Gradle \texttt{init} per una applicazione java:
    \begin{verbatim}
        $ gradle init --type java-application
    \end{verbatim}
    \begin{figure}[H]
    \centering
    \includegraphics[width=0.8\linewidth]{4IntegrationWithOtherTool/tutorial/gradleInit.png}
    \end{figure}
    \item Modificare il build.gradle sostituendolo con questa versione:
    \begin{lstlisting}[frame=single]
plugins {
    id 'java'
    id 'application'
    id 'eclipse'
}

mainClassName = 'App'

repositories {
    mavenCentral()
}

dependencies {
    testImplementation 'junit:junit:4.12'
}

task  wrapper(type: Wrapper) {
    gradleVersion = '4.6'
    distributionType = Wrapper.DistributionType.ALL
}
    \end{lstlisting}
    Con questo setting del file build abbiamo impostato sia il plugin di eclipse sia la distribuzione del wrapper da usare.
    \item Aggiornare quindi il wrapper con la build relativa:
    \begin{verbatim}
        $ ./gradlew wrapper
    \end{verbatim}
    \item Creare i meta dati di eclipse con la build omonima definita dal plugin stesso:
    \begin{verbatim}
        $ ./gradlew eclipse
    \end{verbatim}
    \item Possiamo ora importare la repository su Eclipse:
    \begin{figure}[H]
    \centering
    \includegraphics[width=0.5\linewidth]{4IntegrationWithOtherTool/tutorial/addExistingLocalGitRepository.png}
    \end{figure}
    \item Importare poi il progetto creato:
    \begin{figure}[H]
    \centering
    \includegraphics[width=0.6\linewidth]{4IntegrationWithOtherTool/tutorial/localRepositoryEclipse.png}
    \end{figure}
    \item Effettuare il login su \textbf{\href{https://travis-ci.org}{travis-ci.org}} e aggiungere la repository Github creata:
    \begin{figure}[H]
        \centering
        \includegraphics{4IntegrationWithOtherTool/tutorial/newTravis.png}
    \end{figure}
    Attivare la repository:
    \begin{figure}[H]
        \centering
        \includegraphics{4IntegrationWithOtherTool/tutorial/spuntaTravis.png}
    \end{figure}
    Andare nelle impostazioni della repository:
    \begin{figure}[H]
    \centering
    \includegraphics{4IntegrationWithOtherTool/tutorial/settingsTravis.png}
    \end{figure}
    E indicare al server di eseguire la build solo nel caso in cui ci sia il travis.yml:
    \begin{figure}[H]
    \centering
    \includegraphics[width=0.7\linewidth]{4IntegrationWithOtherTool/tutorial/buildOnlyTravis.png}
    \end{figure}
    \item Creaiamo il file \texttt{travis.yml} nel quale andremo a indicare la build da eseguire, impostiamo quindi il file in questo modo:
    \begin{minted}[gobble=4, frame=single, linenos]{yaml}
    language: java

    jdk:
        - oraclejdk8
        - oraclejdk9

    # cache settings
    before_cache:
        - rm -f  $HOME/.gradle/caches/modules-2/modules-2.lock
        - rm -fr $HOME/.gradle/caches/*/plugin-resolution/
    cache:
        directories:
            - $HOME/.gradle/caches/
            - $HOME/.gradle/wrapper/

    script:
        - ./tutorial/gradlew --no-daemon -b tutorial/build.gradle test
  \end{minted}
  Salvando ed aggiornando potremmo vedere che Travis inizierà ad eseguire la build specificata nel file appena creato. Il risultato dovrebbe essere qualcosa di simile:
  \begin{figure}[H]
    \centering
    \includegraphics[width=0.9\linewidth]{4IntegrationWithOtherTool/tutorial/verifyBuild.png}
  \end{figure}
  \item Possiamo importare il tag dello stato della build nel README della repository per ottenere:
  \begin{figure}[H]
    \centering
    \includegraphics[width=0.8\linewidth]{4IntegrationWithOtherTool/tutorial/tagBuildPass.jpg}
  \end{figure}
  Nella build di travis accanto al nome della repository c'è questo tag:
  \begin{figure}[H]
    \centering
    \includegraphics[width=0.2\linewidth]{4IntegrationWithOtherTool/tutorial/buildPassTag.png}
  \end{figure}
  Cliccandoci si apre una finestra in cui potrete scegliere il branch e la modalità di import, nel nostro caso avremo bisogno del codice markdown che corrisponderà ad una istruzione di questo tipo: 
    \begin{minted}[gobble=4, frame=single, linenos]{yaml}
     [![Build Status]
     (https://travis-ci.org/<User>/<gitRepositoryName>.svg?branch=master)]
     (https://travis-ci.org/<User>/<gitRepositoryName>))
  \end{minted}
  Copiamo e incolliamo (con le dovute modifiche) direttamente nel README.md ottenendo il risultato mostrato sopra.
  \item Prima di proseguire con un esempio pratico importiamo la dipendenza Mockito utile per i test successivi:
  \begin{lstlisting}[frame=single]
dependencies {
    testImplementation 'junit:junit:4.12'
    testImplementation 'org.mockito:mockito-core:2.18.0'
}\end{lstlisting}
    Quando si aggiungono nuove dipendenze è importante indicare a eclipse il nuovo classpath, eseguire quindi sia il task \texttt{dependencies} per scaricare le dipendenze che il task \texttt{eclipse} per aggiornare il classpath.
    \item Nei passi successivi si consiglia di creare un nuovo branch per poter vedere meglio la funzionalità di Travis. Lo schema delle classi da creare è il seguente:
  \begin{figure}[H]
    \centering
    \includegraphics[width=1.0\linewidth]{4IntegrationWithOtherTool/tutorial/classDiagramm.png}
  \end{figure}
  Il codice di test di StudentService è:
  \begin{lstlisting}[frame=single]
import static org.junit.Assert.*;
import static org.mockito.Mockito.*;

import java.util.ArrayList;
import java.util.List;

import org.junit.Before;
import org.junit.Test;

import attsw.exam.example.core.repository.Repository;
import attsw.exam.example.core.service.StudentService;

public class StudentServiceTest {

	StudentService studentService;
	List<Student> studentsList;
	Repository repository;

	@Before
	public void init() {
		studentsList = new ArrayList<Student>();
		repository = mock(Repository.class);
		studentService = new StudentService(repository);
		when(repository.findAll()).thenReturn(studentsList);
	}

	@Test
	public void testAllStudentsWithNoStudents() {
		verifyNumberOfStudents(0);
	}

	@Test
	public void testAllStudents() {
		studentsList.add(newStudentTest("id0"));
		studentsList.add(newStudentTest("id1"));
		verifyNumberOfStudents(2);
	}

	@Test
	public void testOneStudentWhenStudentsIsEmpty() {
		when(repository.findOne("id0")).thenReturn(null);
		assertNull(studentService.oneStudent("id0"));
		verify(repository, times(1)).findOne("id0");
	}

	@Test
	public void testOneStudent() {
		when(repository.findOne("id0")).thenReturn(newStudentTest("id0"));
		Student oneStudent = studentService.oneStudent("id0");
		assertNotNull(oneStudent);
		assertEquals("id0", oneStudent.getId());
		verify(repository,times(1)).findOne("id0");
	}

	private Student newStudentTest(String idStudent) {
		return new Student(idStudent);
	}

	private void verifyNumberOfStudents(int expected) {
		assertEquals(expected, studentService.getAllStudents().size());
		verify(repository, times(1)).findAll();
	}

}
  \end{lstlisting}
  Il codice di test di StudentController è:
   \begin{lstlisting}[frame=single]
import static org.junit.Assert.*;
import static org.mockito.Mockito.*;

import java.util.ArrayList;
import java.util.List;
import java.util.stream.Collectors;

import org.junit.Before;
import org.junit.Test;

import attsw.exam.example.core.controller.StudentController;
import attsw.exam.example.core.service.IStudentService;

public class StudentControllerTest {

	StudentController studentController;
	private List<Student> studentsList;
	private IStudentService iStudentService;

	@Before
	public void init() {
		studentsList = new ArrayList<Student>();
		iStudentService = mock(IStudentService.class);
		when(iStudentService.getAllStudents()).thenReturn(studentsList);
		studentController = new StudentController(iStudentService);
	}

	@Test
	public void testGetAllStudentsWhenThereAreNoStudents() {
		assertGetAllStudent("");
	}

	@Test
	public void testGetAllStudents() {
		studentsList.add(newStudentTest("id0"));
		studentsList.add(newStudentTest("id1"));
		assertGetAllStudent(extractAllStudentsStringFromList(studentsList));
		verify(iStudentService, times(1)).getAllStudents();
	}

	@Test(expected = NullPointerException.class)
	public void testGetOneStudentWhenThereAreNoStudents() {
		assertNull(studentController.getOneStudent("id0"));
	}

	@Test
	public void testGetStudent() {
		Student student = new Student("id0");
		when(iStudentService.oneStudent("id0")).thenReturn(student);
		assertEquals(student.toString(), studentController.getOneStudent("id0"));
		verify(iStudentService, times(1)).oneStudent("id0");
	}

	private String extractAllStudentsStringFromList(List<Student> list) {
		return list.stream().map(student -> student.toString() + System.getProperty("line.separator"))
				.collect(Collectors.joining());
	}

	private void assertGetAllStudent(String expected) {
		assertEquals(expected, studentController.getAllStudents());
		verify(iStudentService, times(1)).getAllStudents();
	}

	private Student newStudentTest(String idStudent) {
		return new Student(idStudent);
	}

}
  \end{lstlisting}
  A partire da queste 2 classi di test è possibile ricavarsi l'implementazione del codice completo:
         \begin{lstlisting}[frame=single]
public class Student {

	private String id;

	public Student(String id) {
		this.id = id;
	}

	public String getId() {
		return this.id;
	}

	@Override
	public String toString() {
		return "Student [id=" + id + "]";
	}
}
\end{lstlisting}
\begin{lstlisting}[frame=single]
import java.util.List;

import attsw.exam.example.core.Student;
import attsw.exam.example.core.repository.Repository;

public class StudentService implements IStudentService{

	private Repository repository;

	public StudentService(Repository repository) {
		this.repository = repository;
	}

	public List<Student> getAllStudents() {
		return repository.findAll();
	}

	public Student oneStudent(String idStudent) {
		return repository.findOne(idStudent);
	}

}
\end{lstlisting}
\begin{lstlisting}[frame=single]
import java.util.stream.Collectors;

import attsw.exam.example.core.service.IStudentService;

public class StudentController implements IStudentController {

	private IStudentService studentService;

	public StudentController(IStudentService iStudentService) {
		this.studentService = iStudentService;

	}

	public String getAllStudents() {
		return studentService.getAllStudents().stream()
				.map(student -> student.toString() + System.getProperty("line.separator"))
				.collect(Collectors.joining());
	}

	public String getOneStudent(String idStudent) {
		return studentService.oneStudent(idStudent).toString();
	}
}
\end{lstlisting}
\begin{lstlisting}[frame=single]
import java.util.List;

import attsw.exam.example.core.Student;

public interface Repository {

	public List<Student> findAll();

	public Student findOne(String id);

}
\end{lstlisting}
  A questo punto si ha un codice effettivo in cui Travis può eseguire la build precedentemente impostata.
  \item Andiamo ora a generare dei report sul codice appena scritto, per farlo aggiungiamo ai plugin di Gradle sia Jacoco che Sonarqube:
    \begin{lstlisting}[frame=single]
plugins {
    id 'java'
    id 'application'
    id 'eclipse'
  	id "org.sonarqube" version "2.6"
  	id 'jacoco'
}

mainClassName = 'App'

repositories {
    mavenCentral()
}

dependencies {
    testImplementation 'junit:junit:4.12'
    testImplementation 'org.mockito:mockito-core:2.18.0'
}

task  wrapper(type: Wrapper) {
    gradleVersion = '4.6'
    distributionType = Wrapper.DistributionType.ALL
}

sonarqube {
    properties {
        property "sonar.projectName", "Java :: tutorial :: SonarQube Scanner for Gradle"
        property "sonar.jacoco.reportPath", "${project.buildDir}/jacoco/"
    }
}
    \end{lstlisting}
    Per eseguire Sonarqube useremo un contenitore server di Docker che genererà una pagina accessibile direttamente dal browser, inoltre Sonarqube avrà bisogno di un database server che creeremo sempre con Docker. Jacoco sarà usato invece da Sonarqube per generare il rapporto di code coverage. Entrambi i server verranno eseguiti compilando un docker-compose, per poter compilare ed eseguire entrambi i server Docker è necessario installare sia docker che docker-compose. All'interno della cartella del progetto creiamo un file chiamato \textbf{docker-compose.yml} in cui scriveremo:
    \begin{minted}[frame=single, linenos]{yaml}
version: "2"

services:
  sonarqube:
    image: sonarqube
    ports:
      - "9000:9000"
    networks:
      - sonarnet
    environment:
      - SONARQUBE_JDBC_URL=jdbc:postgresql://db:5432/sonar
    volumes:
      - sonarqube_conf:/opt/sonarqube/conf
      - sonarqube_data:/opt/sonarqube/data
      - sonarqube_extensions:/opt/sonarqube/extensions
      - sonarqube_bundled-plugins:/opt/sonarqube/lib/bundled-plugins

  db:
    image: postgres
    networks:
      - sonarnet
    environment:
      - POSTGRES_USER=sonar
      - POSTGRES_PASSWORD=sonar
    volumes:
      - postgresql:/var/lib/postgresql
      - postgresql_data:/var/lib/postgresql/data

networks:
  sonarnet:
    driver: bridge

volumes:
  sonarqube_conf:
  sonarqube_data:
  sonarqube_extensions:
  sonarqube_bundled-plugins:
  postgresql:
  postgresql_data:
  \end{minted}
  Questo file è possibile trovarlo nella pagina Github della Docker image di Sonarqube: \href{https://github.com/SonarSource/docker-sonarqube/blob/master/recipes.md}{github.com/SonarSource/docker-sonarqube/blob/master/recipes.md}. Componiamo questo contenitore andando a eseguire il comando:
  \begin{verbatim}
      $ docker-compose up
  \end{verbatim}
  Prenderà un po' di tempo, circa 3/4 minuti in base alla connessione e alla macchina su cui si sta lavorando, avrà finito quando il terminale avrà in output:
  \begin{verbatim}
sonarqube_1  | 2018.04.10 11:49:53 INFO  app[][o.s.a.SchedulerImpl] Process[ce] is up
sonarqube_1  | 2018.04.10 11:49:53 INFO  app[][o.s.a.SchedulerImpl] SonarQube is up
  \end{verbatim}
  A questo punto facciamo ripartire il contenitore, eseguendo in un altro terminale il comando:
  \begin{verbatim}
      $ docker-compose restart sonarqube
  \end{verbatim}
  Una volta completato questo comando possiamo controllare se effettivamente il server sta funzionando visitando la pagina \textbf{\href{http://localhost:9000}{localhost:9000}}:
  \begin{figure}[H]
        \centering
        \includegraphics[width=0.8\linewidth]{4IntegrationWithOtherTool/tutorial/sonarQubeFirst.png}
    \end{figure}
    \item Dobbiamo analizzare il progetto che abbiamo creato, per farlo basterà eseguire il task sonarqube del plugin omonimo, eseguiamo quindi (da terminale o da eclipse):
    \begin{verbatim}
        $ ./gradlew sonarqube
    \end{verbatim}
    Non appena la build sarà completata Sonarqube verrà aggiornato:
    \begin{figure}[H]
        \centering
        \includegraphics[width=0.8\linewidth]{4IntegrationWithOtherTool/tutorial/sonarqubeProjet.png}
    \end{figure}
    Cliccando sul numero di \texttt{Projects Analyzed} verranno mostrati i progetti analizzati e selezionando il progetto da visualizzare avremo la schermata di report del nostro progetto:
    \begin{figure}[H]
        \centering
        \includegraphics[width=0.8\linewidth]{4IntegrationWithOtherTool/tutorial/sonarqubeJacoco.png}
    \end{figure}
    \item Dal terminale possiamo eseguire il comando:
    \begin{verbatim}
        $ docker ps -a
    \end{verbatim}
    in output avremo tutti i contenitori attualmente attivi sulla nostra macchina, nel nostro caso avremo 2 contenitori corrispondenti alle immagini: sonarqube e postgress (database server usato da sonarqube). Una volta completata la nostra analisi del codice è possibile stoppare il contenitore e eliminarlo. Stoppiamo prima il contenitore del database e poi quello di sonarqube:
    \begin{verbatim}
        $ docker stop <nome_del_contenitore>
    \end{verbatim}
    una volta stoppati eliminiamoli usando il comando:
    \begin{verbatim}
        $ docker rm <nome_del_contenitore>
    \end{verbatim}
    \item Potremmo usare Docker anche per creare un server database su cui testare la nostra applicazione, nel nostro caso per creare un vero e proprio database in cui inserire la lista di studenti. Useremo le dipendenze: MongoDB e Jongo per creare il database, FakeMongo per i test. Aggiorniamo quindi il campo \texttt{dependencies} del file di configurazione di gradle:
    \begin{lstlisting}[frame=single]
dependencies {
    testImplementation 'junit:junit:4.12'
    testImplementation 'org.mockito:mockito-core:2.18.0'
    testImplementation 'com.github.fakemongo:fongo:1.6.5'
    testImplementation 'ch.qos.logback:logback-classic:1.1.1'
    implementation 'org.mongodb:mongo-java-driver:2.14.3'
    implementation 'org.jongo:jongo:1.3.0'
}
    \end{lstlisting}
    La dipendenza \texttt{ch.qos.logback:logback-classic:1.1.1} è una dipendenza di FakeMongo necessaria. Da terminale eseguiamo ora i task \texttt{dependencies} per scaricare le dipendenze e \texttt{eclipse} per aggiornare il classpath:
    \begin{verbatim}
        $ ./gradlew dependencies ; ./gradlew eclipse
    \end{verbatim}
    \item Aggiorniamo quindi lo schema delle classi aggiungendo la classe \texttt{MongoRepository} che implementerà la classe \texttt{Repository}:
    \begin{figure}[H]
    \centering
    \includegraphics[width=1.1\linewidth]{4IntegrationWithOtherTool/tutorial/classDiagramm2.png}
  \end{figure}
  Il codice di test per questa nuova classe sarà il seguente:
  \begin{lstlisting}[frame=single]
import static org.junit.Assert.assertEquals;
import static org.junit.Assert.assertNotNull;
import static org.junit.Assert.assertNull;
import static org.junit.Assert.assertTrue;

import java.util.List;

import org.junit.Before;
import org.junit.Test;

import com.mongodb.BasicDBObject;
import com.mongodb.DB;
import com.mongodb.DBCollection;
import com.mongodb.MongoClient;
import com.github.fakemongo.Fongo;
import com.mongodb.MongoClient;

import attsw.exam.example.core.repository.Repository;
import attsw.exam.example.core.repository.mongo.MongoRepository;

public class MongoRepositoryTest {

	Repository mongoRepository;
	DBCollection students;

	@Before
	public void init() {
		MongoClient mongoClient = new Fongo("Mongo Server").getMongo();
		DB db = mongoClient.getDB("School");
		db.getCollection("Students").drop();
		mongoRepository = new MongoRepository(mongoClient);
		students = db.getCollection("Students");
	}

	@Test
	public void testGetAllStudentsEmpty() {
		assertTrue(mongoRepository.findAll().isEmpty());
	}

	@Test
	public void testOneStudent() {
		addStudentToStudentsCollection("id1");
		assertEquals(1, mongoRepository.findAll().size());
	}

	@Test
	public void testMoreThanOneStudentsInCollection() {
		addStudentToStudentsCollection("id1");
		addStudentToStudentsCollection("id2");
		List<Student> listOfStudents = mongoRepository.findAll();
		assertEquals(2, listOfStudents.size());
		assertEquals("id1", listOfStudents.get(0).getId());
		assertEquals("id2", listOfStudents.get(1).getId());
	}

	@Test
	public void testStudentNotFound() {
		assertNull(mongoRepository.findOne("id1"));
	}

	@Test
	public void testStudentFound() {
		addStudentToStudentsCollection("id1");
		addStudentToStudentsCollection("id2");
		Student student = mongoRepository.findOne("id2");
		assertNotNull(student);
		assertEquals("id2", student.getId());
	}

	private void addStudentToStudentsCollection(String idValue) {
		BasicDBObject document = new BasicDBObject();
		document.put("id", idValue);
		students.insert(document);
	}
}
  \end{lstlisting}
  Questa classe di test servirà per la creazione della classe MongoRepository:
  \begin{lstlisting}[frame=single]
import java.util.List;
import java.util.stream.Collectors;
import java.util.stream.StreamSupport;

import com.mongodb.BasicDBObject;
import com.mongodb.DBCollection;
import com.mongodb.DBCursor;
import com.mongodb.DBObject;
import com.mongodb.MongoClient;

import attsw.exam.example.core.Student;
import attsw.exam.example.core.repository.Repository;

public class MongoRepository implements Repository {
	
	DBCollection students;

	public MongoRepository(MongoClient mongoClient) {
		students = mongoClient.getDB("School").getCollection("Students");
	}

	@Override
	public List<Student> findAll() {
		DBCursor cursor = students.find();
		return StreamSupport.stream(cursor.spliterator(), false).map(element -> new Student((String) element.get("id")))
				.collect(Collectors.toList());
	}

	@Override
	public Student findOne(String id) {
		BasicDBObject element = new BasicDBObject();
		element.put("id", id);
		DBObject findOne = students.findOne(element);
		return findOne != null ? new Student((String) findOne.get("id")) : null;
	}

}
  \end{lstlisting}
  I test avranno successo grazie al database fittizio creato da Fongo.
  \item Per poter testare il funzionamento di MongoRepository con un vero database creeremo la classe di Integration Test:
   \begin{lstlisting}[frame=single]
import static org.junit.Assert.assertEquals;
import static org.junit.Assert.assertNotNull;
import static org.junit.Assert.assertNull;
import static org.junit.Assert.assertTrue;

import java.util.List;

import org.junit.Before;
import org.junit.Test;

import com.mongodb.BasicDBObject;
import com.mongodb.DB;
import com.mongodb.DBCollection;
import com.mongodb.MongoClient;
import java.net.UnknownHostException;

import attsw.exam.example.core.repository.Repository;
import attsw.exam.example.core.repository.mongo.MongoRepository;

public class MongoRepositoryIT {

	Repository mongoRepository;
	DBCollection students;

	@Before
	public void init() {
		MongoClient mongoClient = new MongoClient();
		DB db = mongoClient.getDB("School");
		db.getCollection("Students").drop();
		mongoRepository = new MongoRepository(mongoClient);
		students = db.getCollection("Students");
	}

	@Test
	public void testGetAllStudentsEmpty() {
		assertTrue(mongoRepository.findAll().isEmpty());
	}

	@Test
	public void testOneStudent() {
		addStudentToStudentsCollection("id1");
		assertEquals(1, mongoRepository.findAll().size());
	}

	@Test
	public void testMoreThanOneStudentsInCollection() {
		addStudentToStudentsCollection("id1");
		addStudentToStudentsCollection("id2");
		List<Student> listOfStudents = mongoRepository.findAll();
		assertEquals(2, listOfStudents.size());
		assertEquals("id1", listOfStudents.get(0).getId());
		assertEquals("id2", listOfStudents.get(1).getId());
	}

	@Test
	public void testStudentNotFound() {
		assertNull(mongoRepository.findOne("id1"));
	}

	@Test
	public void testStudentFound() {
		addStudentToStudentsCollection("id1");
		addStudentToStudentsCollection("id2");
		Student student = mongoRepository.findOne("id2");
		assertNotNull(student);
		assertEquals("id2", student.getId());
	}

	private void addStudentToStudentsCollection(String idValue) {
		BasicDBObject document = new BasicDBObject();
		document.put("id", idValue);
		students.insert(document);
	}
}
  \end{lstlisting}
  Notiamo che in questa classe di test la differenza è che non useremo Fongo ma un vero e proprio client Mongo. Questo test non potrà funzionare, infatti darà un errore di tipo \texttt{MongoTimeoutException}:
  \begin{figure}[H]
        \centering
        \includegraphics[width=0.8\linewidth]{4IntegrationWithOtherTool/tutorial/mongoDockerFailTest.png}
    \end{figure}
    Che indica semplicemente che la classe di test non riesce a collegarsi al database, che effettivamente non esiste. Prima di creare un contenitore database facciamo un refactor delle 2 classi di test del MongoRepository per rendere più chiara la loro implementazione:
  \begin{lstlisting}[frame=single]
import static org.junit.Assert.assertEquals;
import static org.junit.Assert.assertNotNull;
import static org.junit.Assert.assertNull;
import static org.junit.Assert.assertTrue;

import java.net.UnknownHostException;
import java.util.List;

import org.junit.Before;
import org.junit.Test;

import com.mongodb.BasicDBObject;
import com.mongodb.DB;
import com.mongodb.DBCollection;
import com.mongodb.MongoClient;

import attsw.exam.example.core.repository.Repository;
import attsw.exam.example.core.repository.mongo.MongoRepository;

public abstract class AbstractMongoRepositoryTest {

	Repository mongoRepository;
	DBCollection students;

	@Before
	public void init() throws UnknownHostException {
		MongoClient mongoClient = extractMongoClient();
		DB db = mongoClient.getDB("School");
		db.getCollection("Students").drop();
		mongoRepository = new MongoRepository(mongoClient);
		students = db.getCollection("Students");
	}

	@Test
	public void testGetAllStudentsEmpty() {
		assertTrue(mongoRepository.findAll().isEmpty());
	}

	@Test
	public void testOneStudent() {
		addStudentToStudentsCollection("id1");
		assertEquals(1, mongoRepository.findAll().size());
	}

	@Test
	public void testMoreThanOneStudentsInCollection() {
		addStudentToStudentsCollection("id1");
		addStudentToStudentsCollection("id2");
		List<Student> listOfStudents = mongoRepository.findAll();
		assertEquals(2, listOfStudents.size());
		assertEquals("id1", listOfStudents.get(0).getId());
		assertEquals("id2", listOfStudents.get(1).getId());
	}

	@Test
	public void testStudentNotFound() {
		assertNull(mongoRepository.findOne("id1"));
	}

	@Test
	public void testStudentFound() {
		addStudentToStudentsCollection("id1");
		addStudentToStudentsCollection("id2");
		Student student = mongoRepository.findOne("id2");
		assertNotNull(student);
		assertEquals("id2", student.getId());
	}

	private void addStudentToStudentsCollection(String idValue) {
		BasicDBObject document = new BasicDBObject();
		document.put("id", idValue);
		students.insert(document);
	}


	protected abstract MongoClient extractMongoClient() throws UnknownHostException;
}
  \end{lstlisting}
  Di conseguenza l'implementazione del metodo extractMongoclient() verrà fatta nelle classi effettive di test:
  \begin{lstlisting}[frame=single]
import com.github.fakemongo.Fongo;
import com.mongodb.MongoClient;

public class MongoRepositoryTest extends AbstractMongoRepositoryTest {

	@Override
	protected MongoClient extractMongoClient() {
		return new Fongo("Mongo Server").getMongo();
	}

}
  \end{lstlisting}
    \begin{lstlisting}[frame=single]
import java.net.UnknownHostException;

import com.mongodb.MongoClient;

public class MongoRepositoryIT extends AbstractMongoRepositoryTest {

	@Override
	protected MongoClient extractMongoClient() throws UnknownHostException {
		return new MongoClient();
	}

}
  \end{lstlisting}
  A questo punto possiamo usare un contenitore Docker per creare il nostro database Mongo, per farlo eseguiamo su terminale:
  \begin{verbatim}
        $ docker run -p 27017:27017 --rm mongo
    \end{verbatim}
    Questo comando scaricherà il contenitore chiamato mongo e lo eseguirà sulla porta 27017. A questo punto possiamo eseguire la classe di test MongoRepositoryIT che sfruttando questo contenitore eseguirà con successo i test:
    \begin{figure}[H]
        \centering
        \includegraphics[width=0.8\linewidth]{4IntegrationWithOtherTool/tutorial/mongoDockerSuccessTest.png}
    \end{figure}
    \item Il problema ora è eseguire gli integration test anche su Travis, per farlo useremo la dipendenza \textbf{\href{https://github.com/testcontainers/testcontainers-java}{testcontainers}} che consentirà di creare un containers leggero per il test in cui viene usato. Aggiungiamo quindi la dipendenza \texttt{org.testcontainers:database-commons:1.7.2} al file di build.gradle:
    \begin{lstlisting}[frame=single]
dependencies {
    testImplementation 'junit:junit:4.12'
    testImplementation 'org.mockito:mockito-core:2.18.0'
    testImplementation 'com.github.fakemongo:fongo:1.6.5'
    testImplementation 'ch.qos.logback:logback-classic:1.1.1'
    testImplementation 'org.testcontainers:database-commons:1.7.2'
    implementation 'org.mongodb:mongo-java-driver:2.14.3'
    implementation 'org.jongo:jongo:1.3.0'
}
    \end{lstlisting}
Useremo questa dipendenza nella nuova classe \texttt{MongoRepositoryWithContainerIT}:
    \begin{lstlisting}[frame=single]
import java.net.UnknownHostException;

import org.junit.ClassRule;
import org.testcontainers.containers.GenericContainer;

import com.mongodb.MongoClient;

public class MongoRepositoryWithContainerIT extends AbstractMongoRepositoryTest {

	@SuppressWarnings("rawtypes")
	@ClassRule
	public static GenericContainer mongo = new GenericContainer("mongo:latest").withExposedPorts(27017);

	@Override
	protected MongoClient extractMongoClient() throws UnknownHostException {
		return new MongoClient(mongo.getContainerIpAddress(), mongo.getMappedPort(27017));
	}
}
    \end{lstlisting}
    Come ultimo passo è necessario indicare a travis che per eseguire la build deve installare il servizio docker con database mongo, modifichiamo quindi il .travis.yml come segue:
    \begin{minted}[frame=single, linenos]{yaml}
language: java

jdk:
  - oraclejdk8
  - oraclejdk9

services:
  - docker

# cache settings
before_cache:
  - rm -f  $HOME/.gradle/caches/modules-2/modules-2.lock
  - rm -fr $HOME/.gradle/caches/*/plugin-resolution/

cache:
  directories:
    - $HOME/.gradle/caches/
    - $HOME/.gradle/wrapper/

install:
  - docker pull mongo

script:
- ./tutorial/gradlew --no-daemon -b tutorial/build.gradle test
  \end{minted}
  \item Se la build continuerà a fallire il motivo è che non abbiamo escluso la classe MongoRepositoryIT che necessita il contenitore docker in esecuzione, modifichiamo il file di configurazione build.gradle aggiungendo:
    \begin{minted}[frame=single, linenos]{java}
test {
    failFast = true
    exclude "**/MongoRepositoryIT.*"   
}
  \end{minted}
  L'indicazione di failFast è stata spiegata a pag.\pageref{failfast}. Possiamo anche indicare questa opzione direttamente nello script di travis:
      \begin{minted}[frame=single, linenos]{yaml}
script:
- ./tutorial/gradlew --no-daemon -b tutorial/build.gradle test --fail-fast
  \end{minted}
  A questo punto i test avranno tutti successo anche su travis-ci.
\end{enumerate}

\end{flushleft}
\end{document}
