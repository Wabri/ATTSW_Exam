\subsection{Tutorial}
Il tutorial di seguito è possibile anche trovarlo al link: \href{https://github.com/Wabri/ATTSW_Exam/blob/master/gradle.example/first/}{\textsc{github.com/Wabri/ATTSW\_Exam/blob/master/gradle.example/first/}}.
\begin{enumerate}
    \item Creare una cartella gradle.example/first
    \item All'interno della nuova cartella creare il file build.gradle contenente:
\begin{lstlisting}[frame=single]
description = 'Example of Task'

task dependenceZero {
	description = 'Build Dependence Zero'
	doFirst {
		println 'First Zero'
	}
	doLast {
		println 'Last Zero'
	}
}

task dependenceOne(dependsOn: [dependenceZero]) {
	description = 'Build Dependence One'
	doFirst {
		println 'First One'
	}
	doLast {
		println 'Last One'
	}
}

task dependenceTwo {
	description = 'Build Dependence Two'
	doFirst {
		println 'First Two'
	}
	doLast {
		println 'Last Two'
	}
}

task mainTask(dependsOn: [dependenceOne, dependenceTwo]) {
	description = 'Build Main Task'
	doFirst {
		println 'First MainTask'
	}
	doLast {
		println 'Last MainTask'
	}
}
\end{lstlisting}
    \item Eseguire la build:
\begin{verbatim}
    $ gradle mainTask
\end{verbatim}
    \item  Eseguire la build multi-tasks:
\begin{verbatim}
    $ gradle dependenceZero dependenceTwo
\end{verbatim}
    \item Eseguire la build usando una abbreviazione:
\begin{verbatim}
    $ gradle maTa
\end{verbatim}
    \item Eseguire la build precedente escludendo il task dependenceOne:
\begin{verbatim}
    $ gradle mainTask -x dependenceOne
\end{verbatim}
    \item Creare una build differente in una subdirectory rispetto alla posizione iniziale: 
\begin{lstlisting}[frame=single]
description = 'Sub directory'

task subMainTask {
	description = 'Sub Build Main Task'
	doFirst {
		println 'First MainTask'
	}
	doLast {
		println 'Last MainTask'
	}
}
\end{lstlisting}
    \item Eseguire il task subMainTask della build appena creata partendo dalla directory root:
\begin{verbatim}
    $ gradle -b subdir/build.gradle suMT        
\end{verbatim}
    \item Forzare l'esecuzione di un task marcato come UP-TO-DATE:
\begin{verbatim}
    $ gradle --rerun-tasks maTa
\end{verbatim}
    \item Ottenere la lista dei tasks di default:
\begin{verbatim}
    $ gradle tasks
\end{verbatim}
    \item Ottenere la lista di tutti i tasks:
\begin{verbatim}
    $ gradle tasks --all
\end{verbatim}
    \item Eseguire il comando:
\begin{verbatim}
    $ gradle help --task mainTask
\end{verbatim}
    \item Pubblicare la build del task mainTask:
\begin{verbatim}
    $ gradle mainTask --scan
\end{verbatim}
\end{enumerate}